%!TEX program = xelatex
% 完整编译: xelatex -> bibtex -> xelatex -> xelatex
\documentclass[lang=cn,11pt,a4paper,cite=authoryear]{elegantpaper}

\title{ElegantPaper: 一个优美的 \LaTeX{} 工作论文模板}
\author{Ethan DENG \\ Fudan University \and Dongsheng DENG \\ PA Technology}
\institute{\href{https://elegantlatex.org/}{Elegant\LaTeX{} 项目组}}

\version{0.09}
\date{\zhtoday}


% 本文档命令
\usepackage{array}
\newcommand{\ccr}[1]{\makecell{{\color{#1}\rule{1cm}{1cm}}}}

\begin{document}

\maketitle

\begin{abstract}
本文为 \href{https://github.com/ElegantLaTeX/ElegantPaper/}{ElegantPaper} 的说明文档。此模板基于 \LaTeX{} 的 article 类,专为工作论文写作而设计。设计这个模板的初衷是让作者不用关心工作论文的格式,专心写作,从而有更加舒心的写作体验。如果你有其他问题、建议或者报告 bug,可以在 \href{https://github.com/ElegantLaTeX/ElegantPaper/issues}{GitHub::ElegantPaper/issues} 留言。如果你想了解更多 Elegant\LaTeX{} 项目组设计的模板,请访问 \href{https://github.com/ElegantLaTeX/}{GitHub::ElegantLaTeX}。
\keywords{Elegant\LaTeX{},工作论文,模板}
\end{abstract}



\section{模板使用须知}

\subsection{模板介绍}

此模板基于 \LaTeX{} 的标准文类 article 设计,所以 article 文类的选项也能传递给本模板,比如 \lstinline{a4paper, 11pt} 等等。本模板支持 \hologo{pdfLaTeX} 和 \hologo{XeLaTeX} 编译。

\begin{lstlisting}
\documentclass[a4paper,11pt]{elegantpaper}
\end{lstlisting}

\textbf{注意}:Elegant\LaTeX{} 系列模板已经全部上传至 \href{https://www.overleaf.com/latex/templates/elegantpaper-template/yzghrqjhmmmr}{Overleaf} 上,用户可以在线使用。另外,为了方便国内用户,模板也已经传至\href{https://gitee.com/ElegantLaTeX/ElegantPaper}{码云}。


\subsection{全局选项}
此模板定义了一个语言选项 \lstinline{lang},可以选择英文模式 \lstinline{lang=en}(默认)或者中文模式 \lstinline{lang=cn}。当选择中文模式时,图表的标题引导词以及参考文献,定理引导词等信息会变成中文。你可以通过下面两种方式来选择语言模式:
\begin{lstlisting}
\documentclass[lang=cn]{elegantpaper} % or
\documentclass{cn}{elegantpaper} 
\end{lstlisting}

\textbf{注意:} 英文模式下,由于没有添加中文宏包,无法输入中文。如果需要输入中文,可以通过在导言区引入中文宏包 \lstinline{ctex} 或者加入 \lstinline{xeCJK} 宏包后自行设置字体。 
\begin{lstlisting}
\usepackage[UTF8,scheme=plain]{ctex}
\end{lstlisting}

\subsection{数学字体选项}

本模板定义了一个数学字体选项(\lstinline{math}),可选项有三个:
\begin{enumerate}
  \item \lstinline{math=cm}(默认),使用 \LaTeX{} 默认数学字体(推荐,无需声明);
  \item \lstinline{math=newtx},使用 \lstinline{newtxmath} 设置数学字体(潜在问题比较多)。
  \item \lstinline{math=mtpro2},使用 \lstinline{mtpro2} 宏包设置数学字体,要求用户已经成功安装此宏包。
\end{enumerate}

\subsection{中文字体选项}
模板提供中文字体选项 \lstinline{chinesefont},可选项有
\begin{enumerate}
\item \lstinline{ctexfont}:默认选项,使用 \lstinline{ctex} 宏包根据系统自行选择字体,可能存在字体缺失的问题,更多内容参考 \lstinline{ctex} 宏包\href{https://ctan.org/pkg/ctex}{官方文档}\footnote{可以使用命令提示符,输入 \lstinline{texdoc ctex} 调出本地 \lstinline{ctex} 宏包文档}。
\item \lstinline{founder}:方正字体选项,调用 \lstinline{ctex} 宏包并且使用 \lstinline{fontset=none} 选项,然后设置字体为方正四款免费字体,方正字体下载注意事项见后文。
\item \lstinline{nofont}:调用 \lstinline{ctex} 宏包并且使用 \lstinline{fontset=none} 选项,不设定中文字体,用户可以自行设置中文字体,具体见后文。
\end{enumerate}

\noindent \textbf{注意:} 使用 \lstinline{founder} 选项或者 \lstinline{nofont} 时,必须使用 \hologo{XeLaTeX} 进行编译。

\subsubsection{方正字体选项}
由于使用 \lstinline{ctex} 宏包默认调用系统已有的字体,部分系统字体缺失严重,因此,用户希望能够使用其它字体,我们推荐使用方正字体。方正的{\songti 方正书宋}、{\heiti 方正黑体}、{\kaishu 方正楷体}、{\fangsong 方正仿宋}四款字体均可免费试用,且可用于商业用途。用户可以自行从\href{http://www.foundertype.com/}{方正字体官网}下载此四款字体,在下载的时候请\textbf{务必}注意选择 GBK 字符集,也可以使用 \href{https://www.latexstudio.net/}{\LaTeX{} 工作室}提供的\href{https://pan.baidu.com/s/1BgbQM7LoinY7m8yeP25Y7Q}{方正字体,提取码为:njy9} 进行安装。安装时,{\kaishu Win 10 用户请右键选择为全部用户安装,否则会找不到字体。}

\begin{figure}[!htb]
\centering
\includegraphics[width=0.9\textwidth]{founder.png}
\end{figure}

\subsubsection{其他中文字体}
如果你想完全自定义字体\footnote{这里仍然以方正字体为例。},你可以选择 \lstinline{chinesefont=nofont},然后在导言区设置
\begin{lstlisting}
\setCJKmainfont[BoldFont={FZHei-B01},ItalicFont={FZKai-Z03}]{FZShuSong-Z01}
\setCJKsansfont[BoldFont={FZHei-B01},ItalicFont={FZHei-B01}]{FZHei-B01}
\setCJKmonofont[BoldFont={FZHei-B01},ItalicFont={FZHei-B01}]{FZFangSong-Z02}
\setCJKfamilyfont{zhsong}{FZShuSong-Z01}
\setCJKfamilyfont{zhhei}{FZHei-B01}
\setCJKfamilyfont{zhkai}{FZKai-Z03}
\setCJKfamilyfont{zhfs}{FZFangSong-Z02}
\newcommand*{\songti}{\CJKfamily{zhsong}}
\newcommand*{\heiti}{\CJKfamily{zhhei}}
\newcommand*{\kaishu}{\CJKfamily{zhkai}}
\newcommand*{\fangsong}{\CJKfamily{zhfs}}
\end{lstlisting}


\subsection{自定义命令}
此模板并没有修改任何默认的 \LaTeX{} 命令或者环境\footnote{目的是保证代码的可复用性,请用户关注内容,不要太在意格式,这才是本工作论文模板的意义。}。另外,我自定义了 4 个命令:
\begin{enumerate}
  \item \lstinline{\email}:创建邮箱地址的链接,比如 \email{ddswhu@outlook.com};
  \item \lstinline{\figref}:用法和 \lstinline{\ref} 类似,但是会在插图的标题前添加 <\textbf{图 n}> ;
  \item \lstinline{\tabref}:用法和 \lstinline{\ref} 类似,但是会在表格的标题前添加 <\textbf{表 n}>;
  \item \lstinline{\keywords}:为摘要环境添加关键词。
\end{enumerate}

\subsection{参考文献}
此模板使用 \hologo{BibTeX} 来生成参考文献,中文模式下默认使用的文献样式(bib style)是 \lstinline{GB/T 7714-2015}\footnote{通过调用 \href{https://ctan.org/pkg/gbt7714}{\lstinline{gbt7714}} 宏包}。参考文献示例:~\cite{en3} 使用了中国一个大型的 P2P 平台(人人贷)的数据来检验男性投资者和女性投资者在投资表现上是否有显著差异。

你可以在谷歌学术,Mendeley,Endnote 中获得文献条目(bib item),然后把它们添加到 \lstinline{wpref.bib} 中。在文中引用的时候,引用它们的键值(bib key)即可。注意需要在编译的过程中添加 \hologo{BibTeX} 编译。

本模板还添加了 \lstinline{cite=numbers} 、\lstinline{cite=super} 和 \lstinline{cite=authoryear}  三个参考文献选项,用于设置参考文献格式的设置,默认为 \lstinline{numbers}。理工科类一般使用数字形式 \lstinline{numbers} 或者上标形式 \lstinline{super},而文科类多使用作者-年份 \lstinline{authoryear} 比较多。如果需要改为 \lstinline{cite=numbers}  或者  \lstinline{authoryear} ,可以使用
\begin{lstlisting}
\documentclass[cite=super]{elegantpaper} % super style ref style
\documentclass[super]{elegantpaper}

\documentclass[cite=authoryear]{elegantpaper} % author-year ref style
\documentclass[authoryear]{elegantpaper}
\end{lstlisting}


\section{协作人员招募}
招募 Elegant\LaTeX{} 的协作人员,没有工资。工作内容:翻译 Elegant\LaTeX{} 系列模板相关的文稿(中翻英),维护模板的 wiki(主要涉及 Markdown),如果有公众号文稿写作经历的话,也可以帮忙写微信稿。本公告长期有效。

目前 ElegantLaTeX 共有 4 名协作人员,分别是
\begin{itemize}
  \item 官方文档翻译: \href{https://github.com/peggy2006xzyz}{YPY};
  \item GitHub 维基维护: \href{https://github.com/izinngo}{Ingo Zinngo}、\href{https://github.com/xiaohao890809}{追寻原风景};
  \item QQ 群管理员: \href{https://github.com/sikouhjw}{Sikouhjw}.
\end{itemize}

在此感谢他们无私的奉献!


\section{致谢}
截止到 2020 年 04 月 12 日,ElegantPaper v0.09 版本发布,ElegantPaper 模板在 GitHub 上的收藏数(star)达到了 277。在此特别感谢 China\TeX{} 以及 \href{http://www.latexstudio.net/}{\LaTeX{} 工作室}对于本系列模板的大力宣传与推广。如果你喜欢我们的模板,你可以在 GitHub 上收藏(Star)我们的模板。
\begin{figure}[htbp]
  \centering
  \includegraphics[width=\textwidth]{star.png}
  \caption{一键三连求赞}
\end{figure}

\section{捐赠}
如果您非常喜爱我们的模板,你还可以选择捐赠以表达您对我们模板和我的支持!

\begin{figure}[htbp]
  \centering
  \includegraphics[width=0.5\textwidth]{donate.jpg}
\end{figure}

\textbf{赞赏费用的使用解释权归 Elegant\LaTeX{} 所有,并且不接受监督,请自愿理性打赏}。10 元以上的赞赏,我们将列入捐赠榜,谢谢各位金主!


\begin{table}[!htb]
  \centering
  \caption{Elegant\LaTeX{} 系列模板捐赠榜}
    \begin{tabular}{*{4}{>{\scriptsize}c}|*{4}{>{\scriptsize}c}}
    \hline
    \textbf{捐赠者} & \textbf{金额} & \textbf{时间} & \textbf{渠道} & \textbf{捐赠者} & \textbf{金额} & \textbf{时间} & \textbf{渠道} \\
    \hline
    Lerh  & 10 RMB & 2019/05/15 & 微信    & 越过地平线 & 10 RMB & 2019/05/15 & 微信 \\
    银桑    & 20 RMB & 2019/05/27 & 微信    & *空    & 10 RMB & 2019/05/30 & 微信 \\
    latexstudio.net & 666 RMB & 2019/06/05 & 支付宝   & A*n   & 40 RMB & 2019/06/15 & 微信 \\
    * 夏   & 22 RMB & 2019/06/15 & 微信    & * 倩   & 21 RMB  & 2019/06/15 & 微信 \\
    Cassis & 11 RMB & 2019/06/30 & 微信    & *君    & 10 RMB & 2019/07/23 & 微信 \\
    P*u   & 50 RMB & 2019/07/30 & 微信    & *萌    & 19 RMB & 2019/08/28 & 微信 \\
    曲豆豆   & 10 RMB & 2019/08/28 & 微信    & 李博    & 100 RMB & 2019/10/06 & 微信 \\
    Njustsll & 10 RMB & 2019/10/11 & 微信    & 刘志阔   & 99.99 RMB & 2019/10/15 & 支付宝 \\
    * 韬   & 16 RMB & 2019/10/17 & 微信    & 赤霓    & 12 RMB & 2019/10/17 & 支付宝 \\
    追寻原风景 & 10 RMB & 2019/10/28 & 微信    & 郭德良   & 88 RMB & 2019/11/03 & 微信 \\
    自强不息  & 20 RMB & 2019/11/04 & 支付宝   & 读书之虫  & 20 RMB & 2019/11/18 & 微信 \\
    *等    & 10 RMB & 2019/11/18 & 微信    & *哲    & 20 RMB & 2019/11/18 & 微信 \\
    佚名    & 10 RMB & 2019/11/24 & 微信    & Jiye Qian & 66 RMB & 2019/12/04 & 微信 \\
    * 阳   & 20 RMB & 2019/12/05 & 微信    & Catcher & 11 RMB & 2019/12/08 & 支付宝 \\
    希尔波特门徒 & 10 RMB & 2019/12/09 & 支付宝   & * 伟   & 10 RMB & 2019/12/09 & 微信 \\
    Simon & 20 RMB & 2019/12/11 & 支付宝   & 流殇丶浅忆 & 66.60 RMB & 2019/12/18 & 支付宝 \\
    羽     & 10 RMB & 2019/12/20 & 支付宝   & * 琛   & 15 RMB & 2019/12/20 & 微信 \\
    随风    & 20 RMB & 2019/12/27 & 支付宝   & Ws    & 23.30 RMB & 2019/12/28 & 微信 \\
    初八    & 100 RMB  & 2020/01/02 & 支付宝   & p*e   & 20 RMB & 2020/01/03 & 微信 \\
    Shunmx & 100 RMB & 2020/01/03 & 微信    & hj    & 10 RMB & 2020/01/03 & 微信 \\
    F*5   & 10 RMB & 2020/01/03 & 微信    & S*m   & 20.20 RMB & 2020/01/03 & 微信 \\
    二代青雉  & 13 RMB & 2020/01/14 & 支付宝   & *?    & 66 RMB & 2020/01/15 & 微信 \\
    Mr. Xiong & 20 RMB & 2020/01/17 & 微信    & *博    & 15 RMB & 2020/01/18 & 微信 \\
    * 者  & 10 RMB & 2020/02/02 & 微信    & Jackie  &  88.80 RMB  &  2020/02/09 & 微信 \\
    Henry\_Sun、 & 50 RMB & 2020/02/14 & 支付宝 & * 桥  & 50 RMB & 2020/02/21 & 微信 \\
    昀琏 & 10 RMB & 2020/03/02 & 支付宝 & S*y  &  10 RMB  &  2020/03/15 & 微信 \\
    * 哥  & 66.66 RMB & 2020/03/17 & 微信    &   K*e & 30 RMB & 2020/03/30 & 微信\\
    * 阳  &  20 RMB  &  2020/04/02 & 微信 & 士*n  & 30 RMB & 2020/04/11 & 微信 \\
    \hline
    \end{tabular}%
  \label{tab:donation}%
\end{table}%

\section{常见问题 FAQ}

\begin{enumerate}[label=\arabic*).]
  \item \textit{如何删除版本信息?}\\
      导言区不写 \lstinline|\version{x.xx}| 即可。
  \item \textit{如何删除日期?}\\
      需要注意的是,与版本 \lstinline{\version} 不同的是,导言区不写或注释 \lstinline{\date} 的话,仍然会打印出当日日期,原因是 \lstinline{\date} 有默认参数。如果不需要日期的话,日期可以留空即可,也即 \lstinline|\date{}|。
  \item \textit{如何获得中文日期?}\\
      为了获得中文日期,必须在中文模式下\footnote{英文模式下,由于未加载中文宏包,无法输入中文。},使用 \lstinline|\date{\zhdate{2019/10/11}}|,如果需要当天的汉化日期,可以使用 \lstinline|\date{\zhtoday}|,这两个命令都来源于 \href{https://ctan.org/pkg/zhnumber}{\lstinline{zhnumber}} 宏包。
  \item \textit{如何添加多个作者?}\\
      在 \lstinline{\author} 里面使用 \lstinline{\and},作者单位可以用 \lstinline{\\} 换行。\begin{lstlisting}
\author{author 1\\ org. 1 \and author 2 \\ org. 2 }
\end{lstlisting}
  \item \textit{如何添加中英文摘要?}\\
      请参考 \href{https://github.com/ElegantLaTeX/ElegantPaper/issues/5}{GitHub::ElegantPaper/issues/5}
\end{enumerate}

\section{示例}

为了让大家更加清楚最终的论文效果,如下给出两篇使用 ElegantPaper 模板排版的工作论文示例,也欢迎大家“投稿”!

\begin{enumerate}
  \item \href{https://github.com/EthanDeng/bank-custody}{银行存管、投资者决策与 P2P 网络借贷规范发展};
  \item \href{https://github.com/EthanDeng/risk-awareness}{互联网金融风险与投资者风险意识 —— 来自网贷平台交易数据的证据}。
\end{enumerate}


\nocite{*}
\bibliography{wpref}

\appendix
%\appendixpage
\addappheadtotoc
\section{使用 newtx 系列字体}

如果需要使用原先版本的 \lstinline{newtx} 系列字体,可以通过显示声明数学字体:

\begin{lstlisting}
\documentclass[math=newtx]{elegantbook}
\end{lstlisting}

\subsection{连字符}

如果使用 \lstinline{newtx} 系列字体宏包,需要注意下连字符的问题。
\begin{equation}
  \int_{R^q} f(x,y) dy.\emph{of\kern0pt f}
\end{equation}
的代码为
\begin{lstlisting}
\begin{equation}
  \int_{R^q} f(x,y) dy.\emph{of \kern0pt f}
\end{equation}
\end{lstlisting}

\subsection{宏包冲突}

另外在 ElegantBook 模板中,有用户反馈模板在使用 \lstinline{yhmath} 以及 \lstinline{esvect} 等宏包时会报错:
\begin{lstlisting}
LaTeX Error:
   Too many symbol fonts declared.
\end{lstlisting}

原因是在使用 \lstinline{newtxmath} 宏包时,重新定义了数学字体用于大型操作符,达到了 {\heiti 最多 16 个数学字体} 的上限,在调用其他宏包的时候,无法新增数学字体。为了减少调用非常用宏包,在此给出如何调用 \lstinline{yhmath} 以及 \lstinline{esvect} 宏包的方法。

请在 \lstinline{elegantpaper.cls} 内搜索 \lstinline{yhmath} 或者 \lstinline{esvect},将你所需要的宏包加载语句\textit{取消注释}即可。
\begin{lstlisting}
%%% use yhmath pkg, uncomment following code
% \let\oldwidering\widering
% \let\widering\undefined
% \RequirePackage{yhmath}
% \let\widering\oldwidering

%%% use esvect pkg, uncomment following code
% \RequirePackage{esvect}
\end{lstlisting}


\end{document}
